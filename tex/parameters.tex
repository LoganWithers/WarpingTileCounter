\section{Parameters for the counter}

...therefore, let $d = \floor*{\frac{k}{2}}$, $m = \ceil*{\left(\frac{N}{102}\right)^{\frac{1}{d}}}$,
$l = \ceil{\log m} + 2$, $s = m^d - \floor*{\frac{N - 12l - 94}{\rowheight}}$, where $d$ is the
number of digits per row of the counter, $m$ is the base of the counter, $l$ is the number of bits
needed to encode each digit in binary plus 2 for indicating whether a digit is in the MSR and is the
MSD in that region, and $s$ is the starting value of the counter in decimal.
%
If $d$ is a multiple of 3, let $g = \frac{d}{3} - 1$ otherwise let $g = \floor*{\frac{d}{3}}$, where $g$ is the number of general digit regions.
%


Let $h$ be the height of the construction without any additional ``roof'' tiles.
%
By inspection...we see that the height of the construction without additional tiles is $\seedheight$ for the {\tt Seed} unit, plus $\rowheight$ for each incrementation of the counter, adding a {\tt Counter} unit row each time.
%
If we define $n$ as the number of {\tt Counter} unit rows, then $h = n (\rowheight) + (\seedheight)$.
%
So then the maximum height of the counter is $m^d (\rowheight) + \seedheight$.
%
Since our goal is to end with a rectangle of height $N$, we need to pick a base such that the counter can increment so many times that when it stops, it is at least $N$.

\begin{lemma} $N \leq m^d (\rowheight) + \seedheight$.
    \begin{proof}
        \begin{flalign*}
            N & = 102 \left( \frac{N}{102} \right) = 102 \left(\left( \frac{N}{102} \right)^{\frac{1}{d}} \right)^d
                      \leq 102 \ceil*{\left( \frac{N}{102} \right)^{\frac{1}{d}} }^d \\
              & = 102m^d \leq 12lm^d + 90m^d \leq 12lm^d + 90m^d + \seedheight \\
              & = m^d (\rowheight) + \seedheight
        \end{flalign*}
    \end{proof}
\end{lemma}

And the mininum height is $24l + 184$.

\vspace{1cm}

%
One row of the counter might result in a final assembly that will not be tall enough but ... says that having all possible rows of the counter might result in a final assembly that is too tall.
%
Therefore, we must start the counter at an appropriate value to get the correct height of the final assembly.
%


%
The counter can start at any whole number less than $m^d$ and ends when it reaches 0 by rolling over $m^d - 1$.
%
This means that the number of Counter unit rows $n$, is $m^d - s$, where we have defined $s$ as the starting value of the counter.
%
To choose the best starting value, we find the value for $n$ that gets $h$ close to $N$ without exceeding $N$.
%
It follows from the equation $h = n(\rowheight) + \seedheight$, that $n = \floor*{\frac{N - 12l - 94}{\rowheight}}$.
%
Thus, $s = m^d - \floor*{\frac{N - 12 - 94}{\rowheight}}$.
%

%
If $k$ is an even number, no filler tiles are needed.
%
We can assemble any rectangle that has a width $k$, provided that $k$ is some even number $\geq 6$.
%
If $k$ is any odd number $\geq 6$, we use one filler tiles.
%
...as a result of each digit requiring a width of 2 tiles, if $k$ is odd, one additional tile column must be added.
%
The number of filler tiles needed for the width is $k \mod 2$, and the number of filler tiles for the height is $N - 12l - 90 \mod 12l + 90$.
%

