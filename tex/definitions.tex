\newcommand{\counterstart}{s}
\newcommand{\counterrows}{\floor*{\frac{N-4}{12l + 90}}}
\newcommand{\countersize}{n}

\section{Definitions}
\label{sec:prelims}


\subsection{Misc}

\begin{flalign*}
      \text{Let }  m & = \ceil*{\left( \frac{N}{102} \right)^{\frac{1}{d}} } \text{, base of the counter} & \\
                 MSR & = \text{ most signifcant digit region } \\
                   s & = \text{starting value of counter}  \\
                   d & = \ceil*{\log_m {s}} = \floor*{\frac{k}{2}},  \text{ number of digits per row }  \\
               m^{d} & = \text{ final value of the counter} \\
        \countersize & = m^{d} - s,  \text{ number of rows/ times to count} \\
                   l & = \ceil{\log m} + 2, \text{ bits needed to encode each digit in binary, plus 2 for MSR and MSD} & \\
\end{flalign*}




\subsection{Parameters for the counter}

...therefore, let $d = \floor*{\frac{k}{2}}$, $m = \ceil*{\left(\frac{N}{102}\right)^{\frac{1}{d}}}$,
$l = \ceil{\log m} + 2$, $s = m^d - \floor*{\frac{N - 4}{12l + 90}}$, where $d$ is the
number of digits per row of the counter, $m$ is the base of the counter, $l$ is the number of bits
needed to encode each digit in binary plus 2 for indicating whether a digit is in the MSR and is the
MSD in that region, and $s$ is the starting value of the counter in decimal.
%
%
In general, the height of a digit region is $12l + 90$. There is one exception, being the seed row, which
has a height of $12l + 94$.
%
Let $h$ be the height of the construction without any additional ``roof'' tiles.
%
By inspection...we see that the height of the construction without additional tiles is $12l + 94$ for the {\tt Seed} unit, plus $12l + 90$ for each incrementation of the counter,
adding a {\tt Counter} unit row each time.
%
If we define $\countersize$ as the number of {\tt Counter} unit rows, then $h = (\countersize - 1) (12l + 90) + (12l + 94)$, simplified to $\countersize (12l + 90) + 4$.
So then the maximum height of the counter is $m^d (12l + 90) + 4$.
Since our goal is to end with a rectangle of height $N$, we need to pick a base such that the counter can increment so many times that when it stops, it is at least $N$.

\begin{lemma} $N \leq m^d (12l + 90) + 4$.
    \begin{proof}
        \begin{flalign*}
            N & = 102 \left( \frac{N}{102} \right) = 102 \left(\left( \frac{N}{102} \right)^{\frac{1}{d}} \right)^d
                      \leq 102 \ceil*{\left( \frac{N}{102} \right)^{\frac{1}{d}} }^d \\
              & = 102m^d \leq 12lm^d + 90m^d \leq 12lm^d + 90m^d + 4 \\
              & = m^d (12l + 90) + 4
        \end{flalign*}
    \end{proof}
\end{lemma}


\subsection{Filling in the gaps}

...this means that the number of \texttt{Counter} unit rows $\countersize$ is $m^d - s$,
where we have \mbox{defined} $s$ as the starting value of the counter. To
choose the best starting value, \mbox{ we find the value for $\countersize$ that gets $h$ as
close to $N$} without exceeding $N$. It follows from the equation $h = \countersize (12l + 90) + 4$,
that $\countersize = \counterrows$. Thus, $s = m^d - \counterrows$. As a result of
each digit requiring a width of 2 tiles, if $k$ is odd, one additional tile column must be added.
The number of filler tiles needed for the width is $k \mod 2$, and the number of filler tiles for
the height is $N - 4 \mod 12l + 90$.


