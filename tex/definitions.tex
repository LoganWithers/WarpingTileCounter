\newcommand{\counterstart}{\mathcal{C}_0}
\newcommand{\counterrows}{\floor*{\frac{N-12l-76}{12l + 90}}}
\newcommand{\countersize}{\mathcal{C}_{\Delta}}

\section{Definitions}
\label{sec:prelims}

% \subsection{Digit regions}

%     A digit region is a group of 3 digits within a counter value. Starting with the least significant digit ($d_0$),
%     we consider




\subsection{Misc}

\begin{flalign*}
      \text{Let }  m & = \ceil*{\left( \frac{N}{102} \right)^{\frac{1}{d}} } \text{, base of the counter} & \\
                 MSR & = \text{ most signifcant digit region } \\
       \mathcal{C}_0 & = \text{starting value of counter}  \\
                   d & = \ceil*{\log_m {\mathcal{C}_0}} = \floor*{\frac{k}{2}},  \text{ number of digits per row }  \\
     \mathcal{C}_{f} & = m^{d},                          \text{ final value of the counter} \\
\mathcal{C}_{\Delta} & = \mathcal{C}_f - \mathcal{C}_0,  \text{ number of rows/ times to count} \\
                   l & = \ceil{\log m} + 2, \text{ bits needed to encode each digit in binary, plus 2 for MSR and MSD} & \\
\end{flalign*}




\subsection{Determining the starting value }

...therefore, let $d = \floor*{\frac{k}{2}}$, $m = \ceil*{\left(\frac{N}{102}\right)^{\frac{1}{d}}}$,
$l = \ceil{\log m} + 2$, $\mathcal{C}_0 = m^d - \floor*{\frac{N-12l-76}{12l + 90}}$, where $d$ is the
number of digits per row of the counter, $m$ is the base of the counter, $l$ is the number of bits
needed to encode each digit in binary plus 2 for indicating whether a digit is in the MSR and is the
MSD in that region, and $\mathcal{C}_0$ is the start of the counter in decimal.

In general, the height of a digit region is $12l + 90$. There are two cases when the height is different,
namely in the first and last digit regions, where the height is $12l + 91$ and $12l + 75$, respectively.
Let $h$ be the height of the construction before any filler/roof tiles are added. If we define
$\countersize$ as the number of \texttt{Counter} unit rows, then
$h = (\countersize - 1) (12l + 90) + (12l + 91) + (12l + 75)$, simplifying to $\countersize (12l + 90) + 12l + 76$.
So then the maximum height of the counter is $m^d (12l + 90) + 12l + 76$. Since our goal is to end with a
rectangle of height $N$, we need to pick a base such that the counter can increment so many times that
when it stops, it is at least $N$.

\begin{lemma} $N \leq m^d (12l + 90) + 12l + 76$.
    \begin{proof}
        \begin{flalign*}
            N & = 102 \left( \frac{N}{102} \right) = 102 \left(\left( \frac{N}{102} \right)^{\frac{1}{d}} \right)^d
                \leq 102 \ceil*{\left( \frac{N}{102} \right)^{\frac{1}{d}} }^d \\
            & = 102m^d \leq 12lm^d + 90m^d \leq 12lm^d + 90m^d + 12l + 76 \\
            & = m^d (12l + 90) + 12l + 76
        \end{flalign*}
    \end{proof}
\end{lemma}


\subsection{Filling in the gaps}

...this means that the number of \texttt{Counter} unit rows $\countersize$ is $m^d - \mathcal{C}_0$,
where we have \mbox{defined} $\mathcal{C}_0$ as the starting value of the counter. To
choose the best starting value, \mbox{ we find the value for $\countersize$ that gets $h$ as
close to $N$} without exceeding $N$. It follows from the equation $h = \countersize (12l+90)+12l+76$,
that $\countersize = \counterrows$. Thus, $\mathcal{C}_0 = m^d - \counterrows$. As a result of
each digit requiring a width of 2 tiles, if $k$ is odd, one additional tile column must be added.
The number of filler tiles needed for the width is $k \mod 2$, and the number of filler tiles for
the height is $N - 12l - 76 \mod 12l + 90$.


